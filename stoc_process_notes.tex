\documentclass{beamer}
\usepackage[T1]{fontenc}
\usepackage[utf8]{inputenc}
\usepackage{lmodern}
\usepackage{amsfonts}
\newtheorem{ineq}{Inequality}
\newtheorem{eq}{Equality}
\newtheorem{target}{Target}
\usetheme{Berlin}
\useoutertheme{infolines}
\setbeamertemplate{navigation symbols}{}
\title[Stochastic Process]{Lecture notes of Stochastic Process}
\subtitle{lectured by prof. Hsueh-I Lu}
\author{pishen}
\institute[AlgoLab]{AlgoLab, CSIE, NTU}

\begin{document}
\begin{frame}[plain]
	\titlepage
\end{frame}

\begin{frame}{Thank list}
\begin{center}
LeoSW, windker
\end{center}
\end{frame}

\begin{frame}
	\begin{center}
		{\Huge Limiting Probabilities}
	\end{center}
\end{frame}

\begin{frame}{Outline}
	\begin{enumerate}
		\item Limiting probabilities
		\item Stationary distribution
		\item Long-run proportion
		\item (Inverse of) Expected return time
	\end{enumerate}
\end{frame}

\begin{frame}{Limiting Probabilities}
	\begin{definition}
		Number $\pi_j$ is the \textit{limiting probability} of $j$ if
		\[
		\pi_j = \lim_{n \to \infty} P^n_{ij}
		\]
		holds for all states $i \in S$ ($S \subseteq \mathbb{N}$ is the state space).
	\end{definition}
	\begin{itemize}
		\item $\pi_j$ is independent of $i$.
		\item $\lim_{n \to \infty} P^n = 
			\begin{pmatrix}
				\pi \\
				\pi \\
				\vdots
			\end{pmatrix}$
			, where $\pi = (\pi_1, \pi_2, \ldots)$
	\end{itemize}
\end{frame}

\begin{frame}{Stationary Probability Distribution}
	\begin{definition}
		Non-negative row vector $\pi = (\pi_1, \pi_2, \ldots)$
		is a \textit{stationary probability distribution} of $\mathbb{X}$
		if $\pi \times P = \pi$ holds and $\sum_{i \in S} \pi_i = 1$
	\end{definition}
	\begin{itemize}
		\item $\pi$ is a normalized left eigenvector with eigenvalue $=1$.
		\item If $X(0)$ has distribution $\pi$, then $X(t)$ has the same distribution $\pi$
			for all $t \geq 1$.
			$\pi$ is also called as \textit{steady-state distribution}.
		\item It doesn't mean that each $X(t)$ become independent.
			$\pi$ only means the distribution of $X(t)$ when the previous random variable's value is unknown.
	\end{itemize}
\end{frame}

\begin{frame}{Theorem 1}
	\begin{theorem}
		Let $\mathbb{X}$ be an \textit{irreducible}, \textit{aperiodic}, \textit{positive recurrent} Markov chain, then
		\begin{itemize}
			\item The limiting probability $\pi_j$ of each state $j$ exists.
			\item $\pi = (\pi_1, \pi_2, \ldots)$ is the unique stationary probability distribution.
		\end{itemize}
	\end{theorem}
	\begin{itemize}
		\item The proof will be stated at page \hyperlink{thm1_proof}{\pageref{thm1_proof}}.
	\end{itemize}
\end{frame}

\begin{frame}{Expected return time}
	\begin{definition}
		The \textit{expected return time} of state $i \in S$ is
		\[
		\mu_i = \sum_{n \geq 1} n \cdot f_i^{(n)}
		\]
		where
		\[
		f_i^{(n)} = P(\min\{t:X(t) = i, t\geq 1\} = n | X(0) = i)
		\]
	\end{definition}
	\begin{itemize}
		\item $f_i = \sum_{n \geq 1} f_i^{(n)}$
	\end{itemize}
\end{frame}

\begin{frame}{Positive recurrent \& null recurrent}
	\begin{definition}
		State $i$ is \textit{positive recurrent} if $\mu_i < \infty$
	\end{definition}
	\begin{definition}
		State $i$ is \textit{null recurrent} if $\mu_i = \infty$
	\end{definition}
	\begin{itemize}
		\item Both are recurrent states, and are \textit{class properties}, which means that if state $i$ and $j$ communicate, they will share this property.
		\item If $\mathbb{X}$ is finite, then each recurrent state of $\mathbb{X}$ is positive recurrent.\\
			Proof stated at page \pageref{finite_pos_rec}.
	\end{itemize}
\end{frame}

\begin{frame}{Example of null recurrent}
	\begin{example}
		For a Markov chain with $n$ states ($1,\ldots,n$), if 
		\[
		P(X(t+1)=i+1|X(t)=i) = 1-1/n
		\]
		and
		\[
		P(X(t+1)=1|X(t)=i) = 1/n
		\]
		According to geometric distribution (taking $p = 1/n$), the expectation value of ``steps taken for state 1 to come back'' will be $1/p = n$, hence $\lim_{n\to\infty} n = \infty$.
	\end{example}
\end{frame}

\begin{frame}{Period of a chain}
	\begin{definition}
		The \textit{period} of state $i$ is $d$ if $d$ is the largest integer such that
		\[
		P^n_{ii} = 0
		\]
		holds for all $n$ which is not divisible by $d$.
	\end{definition}
	\begin{definition}
		If each state of $\mathbb{X}$ has period 1, then $\mathbb{X}$ is called \textit{aperiodic}.
	\end{definition}
	\begin{itemize}
		\item If $P_{ii} > 0$ for all $i \in S$, then $\mathbb{X}$ is aperiodic.
		\item Period can be seen as the $\gcd$ of all $n$ that have $P^n_{ii} > 0$, note that $P^{\gcd}_{ii} > 0$ is not necessary.
		\item The period of drunken man problem is 2.
	\end{itemize}
\end{frame}

\begin{frame}{Lemma 1}
	\begin{lemma}
		If state $j$ is aperiodic and positive recurrent, then
		\[
		\pi_j \equiv \lim_{n \to \infty} P^n_{ij}
		\]
		exists and is positive for all states $i \in S$.
	\end{lemma}
	\begin{itemize}
		\item This can be proved by the Blackwell theorem in Renewal theory.
		\item It doesn't promise that each $\pi_j$ for different $i$ will be the same.
			But they will be the same if we add the irreducible property.
	\end{itemize}
\end{frame}

\begin{frame}{Property of $\lim$}
	\begin{itemize}
		\item The position of $\lim$ cannot be switched arbitrarily in an equation.
			\begin{example}
				\[
				1 = \lim_{n \to \infty}\lim_{m \to \infty} \frac{m}{m+n} \neq
				\lim_{m \to \infty}\lim_{n \to \infty} \frac{m}{m+n} = 0
				\]
			\end{example}
		\item $\lim$ would not influence the inequality.
			\begin{example}
				\begin{center}
					If $f(n) \geq g(n)$, then
					$\lim_{n \to \infty} f(n) \geq \lim_{n \to \infty} g(n)$
				\end{center}
			\end{example}
	\end{itemize}
\end{frame}

\begin{frame}{Property of $\lim$ (cont.)}
	\begin{itemize}
		\item $\lim$ is linear operator under finite number of functions.
	\end{itemize}
	\begin{example}
		For $m < \infty$,
		\[
		\sum_{i=1}^m \lim_{n \to \infty} f_i(n) = \lim_{n \to \infty} \sum_{i=1}^m f_i(n)
		\]
	\end{example}
	\textcolor{red}{need an example of $m = \infty$}
%	\begin{itemize}
%		\item It may not work if we push $m$ to $\infty$ first, since $\sum_{i=1}^m f_i(n)$ may influence the result.
%			For example, if $1 \leq i \leq n, f_i(n) = 1/n$, then LHS will be 0, while RHS will be 1.
%			But pushing $m$ to $\infty$ after pushing $n$ to $\infty$ first is OK.
%	\end{itemize}
\end{frame}

\begin{frame}{Inequality 1}
	\begin{ineq}
		\[
		\sum_{j \in S} \pi_j \leq 1
		\]
	\end{ineq}
\end{frame}

\begin{frame}
	\begin{proof}
		\begin{align*}
			\lim_{m \to \infty} \sum_{j=1}^m \pi_j &= \lim_{m \to \infty} \sum_{j=1}^m \lim_{n \to \infty} P^n_{ij} \\
			&= \lim_{m \to \infty} \lim_{n \to \infty} \sum_{j=1}^m P^n_{ij} \\
			&\leq \lim_{m \to \infty} \lim_{n \to \infty} \sum_{j \in S} P^n_{ij} = 1
		\end{align*}
	\end{proof}
	\begin{itemize}
		\item The last equation works since $\sum_{j \in S} P^n_{ij} = 1$.
	\end{itemize}
\end{frame}

\begin{frame}{Inequality 2}
	\begin{ineq}
		For state $j \in S$, we have
		\[
		\pi_j \geq \sum_{i \in S} \pi_i P_{ij}
		\]
	\end{ineq}
\end{frame}

\begin{frame}[shrink]
	\begin{proof}
		For $m \geq 1$ and $n \geq 1$,
		\[
		P^{n+1}_{ij} = \sum_{k \in S} P^n_{ik} P_{kj} \geq \sum_{k=1}^m P^n_{ik} P_{kj}
		\]
		then
		\begin{align*}
			\pi_j &= \lim_{n \to \infty} P^{n+1}_{ij} \\
			&\geq \lim_{n \to \infty} \sum_{k=1}^m P^n_{ik} P_{kj} 
			 = \sum_{k=1}^m \lim_{n \to \infty} P^n_{ik} P_{kj} 
			 = \sum_{k=1}^m \pi_k P_{kj}\\
		\end{align*}
		hence, we know
		\[
		\lim_{m\to\infty} \sum_{k=1}^m \pi_k P_{kj} = \sum_{k \in S} \pi_k P_{kj} \leq \pi_j
		\]
	\end{proof}
\end{frame}

\begin{frame}{Equality 1}
	\begin{eq}
		\[
		\pi_j = \sum_{i \in S} \pi_i P_{ij}
		\]
	\end{eq}
\end{frame}

\begin{frame}
	\begin{proof}
		Assume that for some $j \in S$,
		$\pi_j > \sum_{i \in S} \pi_i P_{ij}$, then
		\begin{align*}
			\sum_{j \in S} \pi_j &> \sum_{j \in S}\sum_{i \in S} \pi_i P_{ij} \\
			&= \sum_{i \in S} \pi_i \sum_{j \in S} P_{ij} = \sum_{i \in S} \pi_i
		\end{align*}
		Since a value cannot be greater than itself, we got contradiction.
	\end{proof}
	\begin{itemize}
		\item $\sum$ should be represented by $\lim \sum$, the equation will still work.
	\end{itemize}
\end{frame}

\begin{frame}{Proof of theorem 1}\label{thm1_proof}
	\begin{itemize}
		\item \textbf{Step 0}: existence of limiting probability.
		\item \textbf{Step 1}: existence of stationary probability distribution.
		\item \textbf{Step 2}: uniqueness.
	\end{itemize}
\end{frame}

\begin{frame}{0. Existence of limiting probability}
	\begin{proof}
		By lemma 1, we know that there exists a $\pi_j$ for row $i$.
		Since the Markov chain is irreducible and all the states are positive recurrent, 
		for any state $i'$ other than $i$, we know that $i'$ surely will visit $i$ in finite steps.
		Therefore, the $\pi_j$ value at row $i'$ will equal to the $\pi_j$ value at row $i$,
		which means that all the $\pi_j$ for column $j$ are the same, and is the limiting probability.
	\end{proof}
	\textcolor{red}{still not clear enough}
\end{frame}

\begin{frame}{1. Existence of stationary probability distribution}
	We want to prove that
	\begin{target}
		There's a vector $s = (s_1, s_2, \ldots)$ such that
		\begin{enumerate}
			\item $\sum_{i \in S} s_i = 1$
			\item $s \times P = s$
		\end{enumerate}
	\end{target}
\end{frame}

\begin{frame}
	\begin{proof}
		By lemma 1, we know that there exists a $\pi = (\pi_1, \pi_2, \ldots)$. \\
		And by equality 1, we know that
		\[
		(\pi_1, \pi_2, \ldots) \times P = (\pi_1, \pi_2, \ldots)
		\]
		Hence $\pi$ can satisfy the 2nd part of our target. \\
		Then, we take $k = \sum_{i \in S} \pi_i$.
		By inequality 1, we know that $k < \infty$, and can get
		\[
		(\frac{\pi_1}{k}, \frac{\pi_2}{k}, \ldots) \times P = (\frac{\pi_1}{k}, \frac{\pi_2}{k}, \ldots)
		\]
		where $\sum_{i \in S} \frac{\pi_i}{k} = 1$ also satisfy the 1st part of our target. \\
		Therefore, this vector can be $s$, which means that it exists.
	\end{proof}
\end{frame}

\begin{frame}{2. Uniqueness}
	\begin{target}
		If $s = (s_1, s_2, \ldots)$ is a stationary distribution of $\mathbb{X}$, then $s = \pi$.
	\end{target}
	\begin{itemize}
		\item We'll prove this by inequality 3 \& 4.
	\end{itemize}
\end{frame}

\begin{frame}{Inequality 3}
	\begin{ineq}
		\[
		s_j \geq \pi_j, \forall j \in S
		\]
	\end{ineq}
\end{frame}

\begin{frame}
	\begin{proof}
		Let the distribution of $X(0)$ be $s$, by the property of stationary distribution, we have
		\begin{align*}
			             s_j &= P(X(n)=j) = \sum_{i \in S} P(X(n) = j | X(0) = i)P(X(0) = i) \\
			                 &= \sum_{i \in S} P^n_{ij} \cdot s_i \\
			                 &\geq \sum_{i=1}^m P^n_{ij} \cdot s_i \\
			\Rightarrow  s_j &= \lim_{m \to \infty}\lim_{n \to \infty} s_j \\
			                 &\geq \lim_{m \to \infty}\lim_{n \to \infty} \sum_{i=1}^m P^n_{ij} \cdot s_i
			                  = \lim_{m \to \infty}\sum_{i=1}^m \pi_j \cdot s_i = \pi_j
		\end{align*}
	\end{proof}
\end{frame}

\begin{frame}{Inequality 4}
	\begin{ineq}
		\[
		s_j \leq \pi_j, \forall j \in S
		\]
	\end{ineq}
\end{frame}

\begin{frame}
	\begin{proof}
		Similar in the proof above, $\forall m,n \geq 1$, we have
		\begin{align*}
			            s_j &= \sum_{i \in S} P^n_{ij} \cdot s_i \\
			                &\leq \sum_{i=1}^m P^n_{ij} \cdot s_i + \sum_{i=m+1}^\infty s_i \\
			\Rightarrow s_j &= \lim_{m \to \infty}\lim_{n \to \infty} s_j \\
			                &\leq \lim_{m \to \infty}\lim_{n \to \infty} \left( 
			                	\sum_{i=1}^m P^n_{ij} \cdot s_i + \sum_{i=m+1}^\infty s_i \right) \\
			                &= \pi_j
		\end{align*}
	\end{proof}
\end{frame}

\begin{frame}{An example Markov chain}
	\begin{example}
		\[
		P = 
		\begin{pmatrix}
			\alpha & 1 - \alpha \\
			\beta  & 1 - \beta
		\end{pmatrix},
		0 < \alpha, \beta < 1
		\]
		\[
		\pi = \left( \frac{\beta}{1+\beta-\alpha}, \frac{1-\alpha}{1+\beta-\alpha} \right) 
		\]
	\end{example}
\end{frame}

\begin{frame}{Real world example: Hardy-Weinberg Law}
	\begin{example}
		There're two kinds of allele: 
		\begin{itemize}
			\item dominant: \textbf{A}
			\item recessive: \textbf{a}
		\end{itemize}
		And three kinds of senotype with population proportion as follow:
		\begin{itemize}
			\item AA: $p$
			\item aa: $q$
			\item Aa: $r = 1 - (p + q)$
		\end{itemize}
	\end{example}
\end{frame}

\begin{frame}
	\begin{example}[cont.]
		\[
		P = 
		\bordermatrix{ ~ & AA                      & aa                      & Aa            \cr
			            AA & p+\frac{r}{2}           & 0                       & q+\frac{r}{2} \cr
			            aa & 0                       & q+\frac{r}{2}           & p+\frac{r}{2} \cr
			            Aa & \frac{p}{2}+\frac{r}{4} & \frac{p}{2}+\frac{r}{4} & \frac{p+q+r}{2} \cr}
		\]
		we get $\pi = (p, q, r)$ when
		\begin{itemize}
			\item $p = {\left( p + \frac{r}{2} \right)}^2$
			\item $q = {\left( q + \frac{r}{2} \right)}^2$
			\item $r = 2 \left( p + \frac{r}{2} \right)\left( q + \frac{r}{2} \right)$
		\end{itemize}
	\end{example}
\end{frame}

\begin{frame}{Long-run proportion}
	\begin{definition}
		We say that $r_j$ is the \textit{long-run proportion} of state $j \in S$ if
		\[
		r_j = \lim_{n\to\infty} \frac{1}{n} \sum_{1 \leq t \leq n} P^t_{ij}
		\]
		holds for each state $i \in S$.
	\end{definition}
	\begin{itemize}
		\item It represents the average appearance times of state $j$ in the whole process.
		\item We will show that (in theorem 3) if $\mathbb{X}$ is irreducible, then the long-run proportion of all states exist.
	\end{itemize}
\end{frame}

\begin{frame}{Theorem 2}
	\begin{theorem}[type 1]
	If $r_j$ exists for each $j \in S$ and $\sum_{j \in S} r_j > 0$,
	then $r = (r_1, r_2, \ldots)$ is the unique stationary distribution of $\mathbb{X}$.
	\end{theorem}
	or
	\begin{theorem}[type 2]
	If $r_j$ exists for each $j \in S$ and \textbf{a stationary distribution exists},
	then $r = (r_1, r_2, \ldots)$ is the unique stationary distribution of $\mathbb{X}$.
	\end{theorem}
\end{frame}

\begin{frame}[shrink]{Proof}
	\textbf{Existence of stationary distribution in type 1:} \\
	Let 
	\[
	R = 
	\begin{pmatrix}
		r \\
		r \\
		\vdots
	\end{pmatrix} 
	= \lim_{n\to\infty} \frac{1}{n} \sum_{1 \leq t \leq n} P^t
	\]
	then
	\begin{align*}
		R \times P &= \lim_{n\to\infty} \frac{1}{n} \sum_{1 \leq t \leq n} P^{t+1} \\
		&= \lim_{n\to\infty} \frac{1}{n} \sum_{1 \leq t \leq n} P^t + \lim_{n\to\infty} \frac{1}{n}(P^{n+1} - P) \\
		&= R
	\end{align*}
	As stated later, $\sum_{j \in S} r_j \leq 1$, hence by normalizing $r$, we prove that stationary distribution exist.
	\begin{itemize}
	\item \textcolor{red}{$(\lim f(n))\cdot g(n) = \lim f(n) \cdot g(n)$?}
	\end{itemize}
\end{frame}

\begin{frame}{Proof (cont.)}
	\textbf{Uniqueness:} \\
	Let $\pi$ be an arbitrary stationary distribution, then
	\begin{align*}
	r &= \pi \times R \\
	&= \pi \times \lim_{n\to\infty} \frac{1}{n} \sum_{1 \leq t \leq n} P^t \\
	&= \lim_{n\to\infty} \frac{1}{n} \sum_{1 \leq t \leq n} \pi \times P^t \\
	&= \lim_{n\to\infty} \frac{1}{n} \sum_{1 \leq t \leq n} \pi \\
	&= \pi
	\end{align*}
\end{frame}

\begin{frame}{Proof (cont.)}\label{proportion_sum}
	\textbf{Prove that $\sum_{j \in S} r_j \leq 1$:}\\
	\begin{align*}
	\sum_{j \in S} r_j & = 
	\lim_{m\to\infty} \sum_{j=1}^m \lim_{n\to\infty} \frac{1}{n} \sum_{t=1}^n P^t_{ij} \\
	& = \lim_{m\to\infty} \lim_{n\to\infty} \frac{1}{n}\sum_{t=1}^n \sum_{j=1}^m P^t_{ij} \\
	& \leq \lim_{m\to\infty} \lim_{n\to\infty} \frac{1}{n}\sum_{t=1}^n \sum_{j\in S} P^t_{ij} \\
	& = \lim_{m\to\infty} \lim_{n\to\infty} \frac{1}{n}\sum_{t=1}^n 1 = 1
	\end{align*}
\end{frame}

\begin{frame}{Example 1}
	On a highway, if we know the probability that
	\begin{itemize}
	\item A truck is followed by a truck: $1/4$
	\item A truck is followed by a car: $3/4$
	\item A car is followed by a truck: $1/5$
	\item A car is followed by a car: $4/5$
	\end{itemize}
	We can construct a matrix
	\[
	\bordermatrix{~ & T   & C   \cr
                  T & 1/4 & 3/4 \cr
                  C & 1/5 & 4/5 \cr}
	\]
	and get the portion of trucks and cars on the whole highway as the eigenvector $(4/19, 15/19)$
	(we will know that long-run proportion exists by Theorem 3).
\end{frame}

\begin{frame}{Example 2}
	For a system which has several good and bad states, we have a matrix $P$:
	\[
	\bordermatrix{~      & g_1 & g_2 & \cdots & b_1 & b_2 & \cdots \cr
                  g_1    &     &     &        &     &     &        \cr
                  g_2    &     &     &        &     &     &        \cr
                  \vdots &     &     &        &     &     &        \cr
                  b_1    &     &     &        &     &     &        \cr
                  b_2    &     &     &        &     &     &        \cr
                  \vdots &     &     &        &     &     &        \cr}
	\]
\end{frame}

\begin{frame}{Example 2 (cont.)}
	\textbf{Q1:} Breakdown rate (breakdown times $/$ total time)\\
	The long-run frequency of going to a bad state from a good state is
	\[
	\sum_{i \in g} \sum_{j \in b} r_i P_{ij}
	\]
\end{frame}

\begin{frame}{Example 2 (cont.)}
	\textbf{Q2:} The expected time $\mu_G$ (resp. $\mu_B$) of staying in good (resp. bad) states once we reach a good (resp. bad) state? \\
	\textbf{Ans:} \\
	For each $t = 1, 2, \ldots$, let $G_t$ (resp. $B_t$) be the length of the $t$-th good (resp. bad) phase of consecutive good (resp. bad) states.
	By the strong law of large numbers, 
	\[
	P\left( \lim_{t\to\infty} \frac{G_1 + B_1 + G_2 + B_2 + \cdots + G_t + B_t}{t} = \mu_G + \mu_B \right) = 1
	\]
	Since the reciprocal of above is the breakdown rate, we get equation (1):
	\[
	P\left( \sum_{i \in G} \sum_{j \in B} \pi_i P_{ij} = \frac{1}{\mu_G + \mu_B} \right) = 1
	\]
\end{frame}

\begin{frame}{Example 2 (cont.)}
	Also, with probability 1, we get equation (2):
	\[
	P\left( \sum_{i \in G} r_i = 
	\lim_{t\to\infty}\frac{G_1 + G_2 + \cdots + G_t}{G_1 + B_1 + \cdots + G_t + B_t}
	= \frac{\mu_G}{\mu_G + \mu_B} \right) = 1
	\]
	Then, by $(2)/(1)$, we get that
	\[
	P\left( \mu_G = \frac{\sum_{i \in G} r_i}{\sum_{i \in G}\sum_{j \in B} r_i P_{ij}} \right) = 1
	\]
	\begin{itemize}
	\item \textcolor{red}{$\lim \frac{f(n)}{g(n)} = \frac{\lim f(n)}{\lim g(n)}$?}
	\end{itemize}
\end{frame}

\begin{frame}{Theorem 3}
	\begin{theorem}
	If $\mathbb{X}$ is irreducible, then the long-run proportion $r_i$ exists with probability 1, moreover,
	\begin{enumerate}
	\item If state $i$ is positive recurrent (i.e. $0 < \mu_i < \infty$), then $P(r_i = \frac{1}{\mu_i}) = 1$.
	\item If state $i$ is null recurrent (i.e. $\mu_i = \infty$) or transient, then $P(r_i = 0) = 1$.
	\end{enumerate}
	\end{theorem}
\end{frame}

\begin{frame}{Proof}
	\textbf{Part 1:} \\
	Suppose $X(0) = i$, $T_k$ is the number of steps required for the $k$-th $i$ goes to $(k+1)$-st $i$,
	then by the strong law of large number,
	\begin{align*}
	&P\left( \lim_{k\to\infty} \frac{T_1 + T_2 + \cdots + T_k}{k} = \mu_i \right) = 1 \\
	\Rightarrow & P\left( r_i = \lim_{k\to\infty} \frac{k}{T_1 + T_2 + \cdots + T_k} = \frac{1}{\mu_i} \right) = 1
	\end{align*}
	\begin{itemize}
	\item \textcolor{red}{$\lim (A/B) = \frac{1}{\lim (B/A)}$?}
	\end{itemize}
\end{frame}

\begin{frame}{Proof (cont.)}
	\textbf{Part 2:} \\
	\begin{enumerate}
	\item If $i$ is transient, $i$ will only appear finite times in the long-run, hence
		\[
		r_i = \frac{finite}{\infty} = 0
		\]
	\item If $i$ is null recurrent, $\mu_i$ is $\infty$, then
		\[
		P\left( \lim_{k\to\infty} \frac{T_1 + T_2 + \cdots + T_k}{k} = \infty \right) = 1
		\]
		\[
		P\left( r_i = \lim_{k\to\infty} \frac{k}{T_1 + T_2 + \cdots + T_k} = 0 \right) = 1
		\]
		(The first equation is not promised by the strong law of large number.
		But if it's not $\infty$, we can say that $\mu_i$ is not $\infty$, which is a contradiction.)
	\end{enumerate}
\end{frame}

\begin{frame}{Example 1}\label{finite_pos_rec}
	\begin{example}[type 1]
	If $\mathbb{X}$ is \textbf{irreducible} and finite, then $\mathbb{X}$ has no null recurrent states.
	\end{example}
	\begin{example}[type 2]
	If $\mathbb{X}$ is finite, then $\mathbb{X}$ has no null recurrent states.
	\end{example}
	\begin{itemize}
	\item Finite irreducible imply positive recurrent.
	\end{itemize}
\end{frame}

\begin{frame}{Proof}
	\begin{itemize}
	\item \textbf{Type 1:}\\
		If there's a state which is null recurrent, by irreducible, all the states will be null recurrent.
		Then, all states have $P(r_i = 0) = 1$.
		By changing the proof in page \pageref{proportion_sum} into finite states version, 
		we know that $\sum r_i = 1$.
		So it's impossible for finite $r_i$, which are all close to 0, to sum up to 1.
	\item \textbf{Type 2:}\\
		If it's not irreducible, the finite set of communicated null recurrent states still form an irreducible and finite Markov chain, which can fit the requirement of type 1.
	\end{itemize}
\end{frame}

\begin{frame}{Example 2}
	\begin{example}
	In the drunken man problem with infinite states, no state will be positive recurrent.
	\end{example}
	\begin{itemize}
	\item Infinite drunken man imply no positive recurrent.
		Note that it doesn't mean all infinite irreducible Markov chain has no positive recurrent state.
	\end{itemize}
\end{frame}

\begin{frame}{Proof}
	If all the states are positive recurrent, then by theorem 3, 
	we know that all the $r_i > 0$ and is a finite value.
	Since each state of drunken man problem has the same structure, all the $r_i$ has same value.
	We then set $r = \epsilon\cdot \min (r_1, r_2, \ldots)$ ($0 < \epsilon < 1$) such that 
	$r_i > r > 0, \forall i$.
	And get
	\[
	\sum_{i\in S} r_i > \sum_{i\in S} r = \infty > 1
	\]
	which is contradiction to \hyperlink{proportion_sum}{page \pageref{proportion_sum}}.
\end{frame}

\begin{frame}{Example 3: Grand Hotel}
	\begin{example}
	There's a hotel, with $N$ representing the number of newly occupied rooms each day ($N$ is a poisson distribution with parameter $\lambda$).
	And the number of consecutive check-in days of each room is a geometric distribution with probability $p$ ($p$ is the probability of check-out).
	$X(t)$ is the number of occupied rooms in day $t$.
	\end{example}
\end{frame}

\begin{frame}{\textbf{Q1:} $P_{ij} = $?}
	We set $R_i$ as a binomial distribution with parameter $(i, 1-p)$, which represents the number of rooms which will remain occupied in the next day, then
	\begin{align*}
	P_{ij} & = P(R_i + N = j) \\
	& = \sum_{k\geq 0} P(R_i + N = j | R_i = k)P(R_i = k) \\
	& = \sum_{k\geq 0} P(N = j-k)P(R_i = k) \\
	& = \sum_{0 \leq k \leq \min(i,j)} \frac{e^{-\lambda}\cdot \lambda^{j-k}}{(j-k)!} \binom{i}{k} (1-p)^k p^{1-k}
	\end{align*}
\end{frame}

\begin{frame}{\textbf{Q2:} $r_i =$?}
	We guess (by a dream?) there's a stationary distribution which is a poisson distribution with parameter $\lambda_0$.
	Setting $X(0)$ with this distribution. 
	And let $R$ as the number of rooms in $X(0)$ which remain check-in in the next day ($R$ is a poisson distribution with parameter $\lambda_0 (1-p)$).
	$X(1)$ will have distribution $R + N$, which is a poisson distribution with parameter $\lambda_0 (1-p) + \lambda$.
	Then since $X(0)$ is a stationary distribution, it will have the same distribution with $X(1)$, which means that $\lambda_0 = \lambda_0 (1-p) + \lambda$, and we get $\lambda_0 = \lambda / p$.
\end{frame}

\end{document}
